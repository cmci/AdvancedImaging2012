%\documentclass[11pnt, twocolumn]{article}
\documentclass[11pnt]{article}
%\documentclass[11pnt]{memoir}

\bibliographystyle{plainnat}
%% Better math support:
\usepackage{amsmath}

%% Bibliography style:
\usepackage{mathptmx}           % Use the Times font.
\usepackage{graphicx}           % Needed for including graphics.
\usepackage{url}                % Facility for activating URLs.

%% Set the paper size to be A4, with a 2cm margin 
%% all around the page.
\usepackage[a4paper,margin=3cm]{geometry}

%% Natbib is a popular style for formatting references.
\usepackage{natbib}
%% bibpunct sets the punctuation used for formatting citations.
\bibpunct{(}{)}{;}{a}{,}{,}

%% textcomp provides extra control sequences for accessing text symbols:
\usepackage{textcomp}
\newcommand*{\micro}{\textmu}
%% Here, we define the \micro command to print a text "mu".
%% "\newcommand" returns an error if "\micro" is already defined.

% header
\usepackage{fancyhdr}
\pagestyle{fancy}
\fancyhead[CO,CE]{EMBL Advanced Imaging 2012}
\fancyhead[RO, LE] {EMBL}
\fancyhead[LO, RE] {Miura}
\renewcommand{\headrulewidth}{0.4pt}
\renewcommand{\footrulewidth}{0.4pt}

%command for menu tree
\newcommand{\ijmenu}[1]{\texttt{\small#1}}
%%%%%%%%%%%%%%%%%%%%%%%%%%%%%%%%%%%%%%%%%%%%%%%%%%%%%%%%%%%%%%%%%%%%%%
%% Start of the document.
%%%%%%%%%%%%%%%%%%%%%%%%%%%%%%%%%%%%%%%%%%%%%%%%%%%%%%%%%%%%%%%%%%%%%%

\begin{document}

\author{Kota Miura\\
  Centre for Molecular and Cellular Imaging\\
  EMBL Heidelberg\\
  69120  Germany\\
  \\
  \texttt{miura@embl.de}\\
  \texttt{http://cmci.embl.de}}
\date{\today}
\title{Tracking Principles / Practicals}
\maketitle
%\begin{abstract}
%Step-by Step Instructions for Directionality Analysis
%\end{abstract}

\begin{abstract}
In the lecture, we will overview methods used for tracking in cell and developmental biology. Focus is made on the details of generally used two-step strategy namely particle detection and linking. In practical works, we use several image processing and analysis tools and explore strategies for quantifying the microtubule polarity of EB1 labeled cells within Drosophila embryo. Students learn how to measure their movement in image sequences by manual tracking, kymograph and automated tracking. 
\end{abstract}

\section{Lecture Notes}

Time series of digital images, usually called ‘a stack’, contains temporal dynamics of position and intensity. By analyzing these dynamics, we can extract numerical parameter which then enables us to characterize the biological system. There are three types of dynamics. 
\begin{enumerate}
\item Position does not change but intensity changes over time. 
\item Position changes but the intensity does not change. 
\item Both Position and Intensity change over time. 
\end{enumerate}
The lecture is about the measurement of type 2, generally called object tracking. Realistic situation demands type 3 \textit{e.g.} you need to get a FRAP curve out of moving organelle, but we approximate the sample as type 2 as we need to learn the principle of tracking first, and type 3 will be your research that asks your ability to apply the full knowledge you acquired in the EMBL Advanced Imaging Course 2012, and probably some technique to use scripting languages.  
\\
\\
For recent review on object tracking in cell and developmental biology, see \cite{Meijering2012}. This article also contains an extensive list of available tracking packages. A bit old review on single particle tracking by Saxton is still an excellent textbook affording rich biophysical insights for analyzing tracking results \citep{Saxton1997}. For the classic ``tracking'' mentioned in the talk, or the presumptive fate map, see \cite{Vogt1925} only if you love biology (I am sure you do). This historical article could be downloaded via Google books. Magnificent movies produced by Keller et. al. using single plane illumination microscopy and extensive image processing that almost killed EMBL cluster by steaming it could be accessed through the web page of the paper in Science \citep{Keller2008}. Comparison of segmentation strategy in object tracking was published in 2001 by \cite{Cheezum2001a} and has been a frequently cited paper in the field of biological object tracking. For overall review on image analysis methods see \cite{Hamilton2009}.


\subsection{Segmentation techniques}
\begin{itemize}
\item Manual tracking\\
ImageJ ``manual tracker plugin'', we will do this in the practical.
\item Manual contour tracing and centroid
\item Thresholding
\item Local intensity maxima
\item Gaussian-fitting
\begin{itemize}
\item Many applications for getting sub-pixel resolution coordinates.
\item You also saw this already in super-resolution microscopy talks by Ricardo and Melike. 
\item Vaccinia Virus tracking example
\end{itemize}
\item Active Contour (SNAKES), Level-set
\begin{itemize} 
\item \url{http://iacl.ece.jhu.edu/projects/gvf/}
\item Quimp11, ImageJ Plugin\\ \url{http://www2.warwick.ac.uk/fac/sci/systemsbiology/staff/bretschneider/quimp/}
\end{itemize}
\item Machine Learning
\begin{itemize}
\item There is no tracking package that does tracking using machine learning based segmentation, but it is a strong candidate in coming days. 
\item Fiji plugin ``Advanced Weka Segmentation''\\
\url{http://fiji.sc/Advanced_Weka_Segmentation}
\end{itemize}
\item Pattern matching (cross-correlation or sum-of-difference technique)
\begin{itemize}
\item 3D tracking of macrophage-like cells. \\See \cite{Grabher2007}
\end{itemize}
\end{itemize}

\subsection{Linking techniques}

\begin{itemize}
	\item Nearest-Neighbor Linking\\
	This is the simplest principle linking an object in one time point to the other. 
	\item Evaluation of Overlaps
	\item Graph-theory based global optimization
	\begin{itemize}
		\item \cite{Vallotton2003a}
	\end{itemize}
	\item Designing cost functions for the global optimization. 
	\begin{itemize}
		\item \cite{Sbalzarini2005a}
		\item \cite{Sbalzarini2006a}
	\end{itemize}
	\item ``Tracklets'', a new way of dealing with dividing cells.
	\begin{itemize}
		\item \cite{Jaqaman2009a}
		\item \cite{Bise2011}
	\end{itemize}
\end{itemize}

\subsection{Tracking Packages we do not try, but worth mentioning here.}

For single molecule trackers, there is a pretty new plugin called PTAj, written by Yoshiyuki Arai. We do not try this plugin in the practical but for those involved in single molecule imaging, it might be worth trying it at your home institute. The plugin offers sophisticated interface to review the tracking results. We all know that even though the tracking is automatic, we start checking each track manually just to be sure, and just to be sure\dots

\begin{itemize}
\item \url{https://github.com/arayoshipta/projectPTAj}
\end{itemize}

There is also a new ``all-in-one'' tracking plugin for ImageJ. It's name is TrackMate, written by Jean-Yves Tinevez. The linking algorithm includes the most-up-to-date ``Tracklets'' and allows to track merging/splitting objects. I have had not time to test this plugin thoroughly, but the interface is similar to Imaris and should be easy for people who are used to wizard-style parameter setting for tracking. For developers, the attractive feature of this plugin is that it offers interface to implement your own algorithm for segmentation and linking. 

\begin{itemize}
\item \url{http://fiji.sc/TrackMate}
\end{itemize}

\section{Sample Sequences}

\begin{description}

\item[Get Image Files]\hfill\\

If capturing of EB3 image sequences were successful (I hope so!!), we use them. Otherwise, we use an example image sequence of EB1 labeled cells. EB1 behaves similar to EB1 and moves along microtubules. A file named \textbf{eb18\_b.tif} is a sequence taken from single cultured cell labeled with eb1. This sequence is accessible by downloading an ImageJ plugin "Sample Image Loader". Instruction for installing this plugin is written in below.  

Download and save the image files, the sequence you have taken or the example sequence eb18\_b.tif in your local computer. We try several different ways to measure movement of EB3 or EB1 protein by tracking. Before tracking, examine the sequence first using stack related functions. 

\item[Open the file in Fiji by]\hfill\\

\verb"[File > Open...]"\\

\item[Check dimensions of the image by]\hfill\\
\verb"[Image > Properties...]"\\

\dots Image stacks are by default taken as a z-series and not t-series. Set Slices to 1, and Frames to appropriate size (number of frames).

\item[Examine the sequence using stack tools.]\hfill\\

Explore stack functions. \\
Start animation, Stop animation, change frame rates, Manipulations\dots These functions are located under\hfill\\

\verb"[Image > Stack >]"\\

\item[A useful plugin for stack]\hfill\\

Running Z Projector plugin, written by Nico Stuurman, averages defined number of frames with moving window. This processing is powerful when you want to reduce noise from time-lapse sequences. Drawback is that some details of movement might be attenuated due to averaging. The plugin could be downloaded from
\begin{itemize}
\item\url{http://valelab.ucsf.edu/~nico/IJplugins/Running_ZProjector.html}
\end{itemize}
\begin{enumerate}
\item Download the plugin from above URL.
\item Install the plugin using \verb"[Plugin > Install...]"
\item Restart Fiji, just to be sure with the installation. 
\item Try running it on EB1 or EB3 sequence. 
\end{enumerate}

\item[A plugin for sample images]\hfill\\

Download the plugin "Sample Image Loader" linked in the following page and install just like you did with the running Z-projector. 
\begin{itemize}
\item\url{http://cmci.embl.de/downloads/sampleimageloader}
\end{itemize}

\end{description}

\section{Manual Tracking}

In the simplest case, tracking can be done manually. The user can read
out the coordinate position of the target object directly from the
imaging software. In ImageJ, user can read out position coordinate
indicated in the status bar by placing the cross-hair pointer over the
object. Then coordinates can be listed in standard spreadsheet software
such as Microsoft Excel or R for further analysis. 

An ImageJ plug-in is also freely available to assist such simple way of
tracking. An obvious disadvantage of the manual tracking is that the
mouse-clicking by the user could be erroneous, as we are still human
who gets tired after thousands of clicking. For such errors,
measurement errors can be estimated by tracking the same object several
times and this error could then be indicated together with the results.
Otherwise, automated tracking is more objective, but if you have only
limited number of tracks to analyze, manual tracking is a best choice
before start to explore complex parameter space of automated tracking
setting.

The manual tracking can be assisted by an ImageJ plug-in
"Manual Tracking". This plug-in
enables the user to record x-y coordinates of the position where the
user clicked using mouse in each frame within a stack. The download
site has a detailed instruction on how to use.  

\begin{description}
\item[Download and Installation]\hfill\\

\textbf{Manual Tracking (Manual\_Tracking.class)} is a plugin bundled with Fiji. If you are using ImageJ, download the plugin file from the link below. The plugin was written by Fabrice Cordelieres.
\begin{itemize}
\item \url{http://rsbweb.nih.gov/ij/plugins/track/track.html}
\end{itemize}

\item[Measurements]\hfill\\

\begin{enumerate}
\item Your task now is to track the movement of individual EB3 signal using the ManualTracker PlugIn. To ease detection of the signal, enhance contrast and convert the image stack from 16-bit to 8-bit.
\\
\\
\verb"[Image > Adjust > Brightness and Contrast]"
\\
\\
and then convert to 8-bit
\\
\\
\verb"[Image > Type > 8bit]"
\\
\item Then 
\begin{enumerate}
\item Check "centering correction", use Local Maximum.
\item Start manual tracking by clicking "Add track".
\item End tracking by "End Track"
\item Show tracks by "Drawing" Functions"
\end{enumerate}

You could track different particles by repeating steps between 2 and 3. Results window will list the measured positions for particles and step 4 will show a track overlaid image stack.

\item Tracked data can be saved by activating "Result" window and 
\\
\\
\verb"[File > Save as...]"
\\
\item \textbf{OPTIONAL:} Copy and paste the result table and plot the track in Excel or R and calculate a histogram of velocity distribution.
\end{enumerate}

\end{description}

\section{Kymograph}

Kymographs are a two-dimensional time traces, where time \textit{t} is in Y axis and space along a one-dimensional contour is in \textit{X}, and the dynamical variable \textit{F(x,t) }is
visualized as an image. Kymographs provide a fast and convenient way to
visualize motion and dynamics in microscopy images.We try measuring the speed of EB1 or EB3 movement using Kymographs. 

\begin{enumerate}
\item Open the image stack. 
\item Contrast enhance, 

\verb"[Image > Adjust > Brightness and Contrast]"

and then convert to 8-bit

\verb"[Image > Type > 8bit]"

\item Do maximum intensity Z-projection 

\verb"[Image > Stacks > Z-projection]". 

\item Choose segmented line ROI tool (to do this, right click line-ROI icon in the menu bar. You will see a drop down selection). With projection image generated above, trace one of the tracks in the projection image.

\item Go back to the stack (click the window and activate it - or bring it to the front - ), 

\verb"[Edit > Selection > Restore Selection]"

Then 

\verb"[Image > Stacks > Reslice...]" 

Don't change parameters in the dialog window, simply OK. Resulting image is the kymograph. 

\item To estimate the velocity of movement from kymograph, you simply need to measure the slope of the diagonal line shown in the kymograph. Since x-axis is distance and y-axis is time, if the starting point of the diagonal line is $(x_1, y_1)$ and the end point of the line is $(x_2, y_2)$ then the $slope = velocity$ is  
\[
velocity = (x_2 - x_1) / (y_2 - y_1) 
\]
Here is a simple macro to calculate this automatically from straight line selection. 
\begin{description}
\item ImageJ macro for measuring kymograph
\begin{verbatim}
// scale and dt must be manually set
var xyscale=0.207; //um
var dt=0.69; //sec

// sampling ROI coordinates, should be a straight line selection
print("-------------------");
getLine(x1, y1, x2, y2, lineWidth);
print("start ("+x1+" , "+y1+")   end ("+ x2 + " , " + y2 + ") ");

// calculation of speed. 
dx=abs(x2-x1);
dy=abs(y2-y1);
dx *= xyscale;
dy *= dt;
velocity= dx/dy;
//output in log window.
print("dx = "+ dx +"um   during:" + dy + "sec");
print("Velocity = "+ d2s(velocity, 3) +" [um/s]");
\end{verbatim}
\end{description}

To use this macro, do 
\\
\\
\verb"[File > New > Script]"
\\
to launch script editor, then select 
\\
\\
\verb"[Language > ImageJ Macro]" 
\\
in the script editor menu, copy and paste the code above. You could also copy and paste from the following page. 
\begin{itemize}
\item \url{http://cmci.embl.de/documents/121005advancedimg}
\end{itemize}

\item \textbf{OPTIONAL:} Collect your data in Excel or R and plot tracks and a histogram of velocity distribution.


\item \textbf{OPTIONAL:} For those interested in using kymograph intensively, there is a good plugin that assists you in doing so efficiently. The plugin was written by Jens Rietdorf and Arne Seitz
\begin{itemize}
\item Multiple Kymograph Plugin for ImageJ\\  \url{http://www.embl.de/eamnet/html/kymograph.html}
\end{itemize}

\item \textbf{OPTIONAL:} To quantify ambiguous kymograph (meaning hard to draw a line, texture like streakes) use:
\begin{itemize}
\item Kymoquant macro for ImageJ\\ \url{http://cmci.embl.de/downloads/kymoquant}
\end{itemize}
Written by myself, with valuable suggestions from Peter Lenert. 

\end{enumerate}

\section{Automated Tracking: ParticleTracker Plugin}

This plugin uses linking algorithm that involves global optimization based on cost function that involves signal intensity and its second order derivative. The plugin was first written as a Matlab code by Ivo Sbalzarini and the migrated to ImageJ plugin by his group members, Guy Levy and Janick Cardinale.

\begin{description}

\item[Download ParticleTracker Plugin and Install]\hfill\\

The plugin could be down loaded from
\begin{itemize}
\item \url{http://www.mosaic.ethz.ch/Downloads/ParticleTracker}
\end{itemize}
To install the downloaded plugin, \verb"[Plugins > Install...]". You could also directly copy the file to plugins folder under Fiji directory and do \verb"[Help > Refresh Menus]".

Just in case if the above plugin somehow does not work, try download and use the version linked in the flowing site. 

\begin{itemize}
\item \url{http://cmci.embl.de/downloads/particletracker2d}
\end{itemize}

\item[Start the ParticleTracker plugin]\hfill\\

Start the particleTracker by \verb"[Plugins > Mosaic > ParticleTracker 2D/3D]". 

\item[Study Dot Detection Parameter]\hfill\\

This tracking tool has two parts. First, all dots in each frame are detected, and then dots in successive frames are linked. The first task then is to determine three parameters for dot detection. There are three parameters. 
\begin{enumerate}
\item \textbf{Radius}\hfill\\
Expected diameter of dot to be detected in pixels. 
\item \textbf{CutOff}\hfill\\
Cutoff level for the none-particle discrimination criteria, a value for each dot that is based on intensity moment order 0 and 2. 
\item \textbf{Percentile}\hfill\\
Larger the value, more particles with dark intensity will be detected. Corresponds to the area proportion below intensity histogram in the upper part of the histogram. 
\end{enumerate}
Try setting different numbers for these parameters and click "Preview Detected". Red circles appear in the image stack. You could change the frames using the slider below the button. 

After some trials, set parameters to what you think is optimum. 

\item[Set Linking parameters]\hfill\\

Two parameters for linking detected dots should be set. 
\begin{enumerate}
\item \textbf{Link Range}\hfill\\
\dots could be more than 1, if you want to link dots that disappears and reappears. If not, set the value to 1.
\item \textbf{Displacement}\hfill\\
\dots expected maximum distance that dots could move from one frame to the next. Unit is in pixels. 
\end{enumerate}
After parameters are set, click "OK". Tracking starts. 

\item[Inspect the Tracking Results]\hfill\\

When tracking is done, a new window titled "Results" appears. At the bottom of the window, there are many buttons. Click "Visualize all trajectories", and then a duplicate of the image stack overlaid with trajectories will appear. 

Select a region within the stack using rectangular ROI tool and then click "Focus on Area". This will create another image stack, with only that region. Since this image is zoomed, you could carefully check if the tracking was successful or not. 

If you think the tracking was not successful, then you should reset all the parameters and do the tracking again.

\item[Export the tracking results]\hfill\\

To analyze the results in R, data should be saved as a file. To do so, first click "All Trajectories to Table". Results table will then created. In this results table window, select \verb"[File > Save As...]" and save the file on your desktop. By default, file type extension is ".xls", excel format, but change this to ".csv". CSV stands for "comma separated file", and this is more classic but general data format which you could easily import in many software including R.  


\item [Further analysis] \hfill\\

\textbf{OPTIONAL:} Import your results from Excel or R and plot tracks and a histogram of velocity distribution.

\end{description}

\section{Particle Image Velocimetry (PIV)}

\begin{description}
\item[Download and Installation]\hfill\\

There is a \textbf{PIV} plugin bundled with Fiji, but we use another PIV plugin developed by Qingzong Tseng. You need several files for this plugin to run. Here is the recipe for running the plugin in 64-bit Windows OS. 
 
\begin{enumerate}
\item Go to 
\begin{itemize}
\item\url{https://sites.google.com/site/qingzongtseng/piv} 
\end{itemize}
and scroll the page down to ``Installation'' section. There you will find links to \textbf{PIV\_.jar}, \textbf{javacv.jar} and \textbf{jna.jar} file. Download these three files.

\item Move or copy above three files to the \textbf{plugin folder} within Fiji or ImageJ folder. 

\item Go to 
\begin{itemize}
\item\url{https://sites.google.com/site/qingzongtseng/template-matching-ij-plugin} 
\end{itemize}
(or you could click ``template matching and slice alignment'' link in the side bar in Zong's page) and scroll down to ``Installation'' section. There you will find a link to OpenCV library for 64-bit windows. Download the file and unzip it. You will find two files, cv100.dll and cxcore100.dll, then. 

\item Move or copy above two files, cv100.dll and cxcore100.dll, to the Fiji or ImageJ folder (\textbf{not the plugin folder}).

\end{enumerate}
\item[Analysis]\hfill\\
\begin{enumerate}
\item PIV plugin works only with two-frame stack. Reduce the stack size by Duplicating only part of the stack. To do so:\\
\begin{itemize}
\item \verb"[Image > Duplicate...]" then in the dialog panel, set the duplication range only two frames e.g. 10-11. 
\end{itemize} 
\item There are three different functions for doing PIV under \verb"[Plugins > PIV >]", Cross-Correlation, Basic and Advanced. Zong recommends not to use Cross-Correlation. We try to do PIV using ``Basic''.
\item When you choose \verb"[Plugins > PIV > Basic]", you will see a panel asking you to input parameters. There are three iterations, first start to estimate displacement of signal with large grids and then to finer grids. By starting from larger grids, estimation procedure recognizes global movement first and then more detailed movements in later iterations. Try to set windows size so that it fits the size of the image. 
\item Final output is a text file. This could be saved anywhere, but be sure that you know where it is. 
\item To plot the PIV field (Vectors), \verb"[Plugins > PIV > Plot]". Try to plot without changing the default value. If the plot looks bad with too large / small arrows, close the plot and do the plotting again with sums adjusted parameters.  
\item \textbf{OPTIONAL} Since output is a text file of vector values, you could import the text file from Excel or R to plot histogram of velocities. 
\end{enumerate}

\bibliography{20121005AdvancedImaging_kota.bib}

\end{description}


\end{document}